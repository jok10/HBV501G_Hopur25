\documentclass{article}
\usepackage{booktabs}
\usepackage{longtable}
\usepackage{array}
\usepackage{geometry}
\geometry{a4paper, margin=1in}

\title{Assignment 1 - Vision and Scope Document}
\author{Team [Number]}
\date{\today}

\begin{document}

\maketitle

\section{Project Vision}

% Provide your project vision document here

\subsection{1.5 Vision Statement} 

\subsection{1.x Additional Section Title (except 1.1)} 

\subsection{2.x Additional Section Title (except 2.1)}

\section{Use Case Document}

\subsection{Add Songs to a Playlist [UC1]}
\textbf{Primary Actor:} User

\textbf{Preconditions:} 
\begin{itemize}
  \item The user must be logged into the system.
  \item The user must have a playlist that has already been created.
\end{itemize}

\textbf{Success Guarantee:}  
The song selected by the user is added to the playlist.

\textbf{Main Success Scenario:}  
The user opens a playlist that already exists. They select the ``add song'' option. They search for a song (see UC4) and select it. The system adds the song to the playlist and confirms successful addition.

\textbf{Extensions / Alternate Scenarios:}  
If the song is already in the playlist, the system displays ``This song has already been added to the playlist.''

\textbf{Miscellaneous / Open Issues:}  
Possibility to add multiple songs at once.

\subsection{Search Song [UC2]}
\textbf{Primary Actor:} User

\textbf{Preconditions:}
\begin{itemize}
  \item None
\end{itemize}

\textbf{Success Guarantee:}  
The system returns relevant songs matching the user’s search criteria.

\textbf{Main Success Scenario:}
\begin{enumerate}
  \item The user selects the ``search'' option.
  \item The user enters a search query (song name, artist, or both).
  \item The system searches the music database.
  \item The system displays a list of matching songs.
  \item The user can view details of a selected song.
\end{enumerate}

\textbf{Extensions / Alternate Scenarios:}
\begin{itemize}
  \item If no results are found, the system displays ``No songs found.''
  \item If the search query is empty, the system prompts the user to enter a valid search term.
\end{itemize}

\textbf{Miscellaneous / Open Issues:}  
Consider supporting advanced search filters (e.g., album, genre, release year).

\subsection{Brief Use Cases}

\begin{itemize}
    \item{\textbf{Brief Use Case 1 [UC5]}: The user is able to generate a link to share their playlist with friends, family, and the community.}
    \item{\textbf{Brief Use Case 2 [UC6]}: ...}
\end{itemize}

\section{Project Estimation and Prioritization}

Below is a table with prioritized use cases:

\begin{table}[h]
  \centering
  \begin{tabular}{|c|c|c|}
    \hline
    Use Case & Time Estimation & Priority \\ \hline
    UC1      & 10             & P1       \\ \hline
    UC4      & 12             & P1       \\ \hline
    UC5      & 8              & P2       \\ \hline
  \end{tabular}
\end{table}

\section{Project Plan and Schedule}

\begin{longtable}{|c|c|c|c|c|c|}
    \hline
    \textbf{Week} & \textbf{Use Cases} & \textbf{Expected Hours} & \textbf{P.O. (Initials)} & \textbf{Sprint} & \textbf{Consultation}\\
    \hline
    1 & None & XX & AB & 1 & \textbf{A1 Presentation}\\
    \hline
    2 & UC1, Android skeleton & XX & AB & 1 & Model Drafts\\
    \hline
    3 & UC4, UC5 & XX & CD & 2 & \textbf{A2 Presentation}\\
    \hline
    4 & UC6, UC7 & XX & CD & 2 & Dev support\\
    \hline
    5 & UC8, UC9 & XX & EF & 3 & \textbf{A3 Presentation}\\
    \hline
\end{longtable}

\section{Project skeleton}
Link to your Github. Indicate who owns which account.  
Ensure everyone performs a small commit.

\end{document}
