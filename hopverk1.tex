\documentclass{article}
\usepackage{booktabs}
\usepackage{longtable}
\usepackage{array}
\usepackage{geometry}

\usepackage[T1]{fontenc}
\usepackage[utf8]{inputenc}
\usepackage{lmodern}
\usepackage[icelandic,english]{babel}

\geometry{a4paper, margin=1in}
\usepackage[hidelinks]{hyperref} 




\geometry{a4paper, margin=1in}

\title{Assignment 1 - Vision and Scope Document}
\author{Team [25]}
\date{\today}

\begin{document}

\maketitle

\section{Project Vision}

\subsection{1.3 Business Objectives}
\begin{itemize}
    \item Users can create, delete and name their playlists, as well as set their visibility to others.
    \item Users can add start and end markers to individual tracks, allowing them to play only the parts they want, skipping unwanted sections.
    \item The order of tracks in each playlist remains stable after edits or deletion of the actual tracks.
    \item Playlists can be tagged, searched and filtered, helping users find the content they want.
    \item Users can share their playlists via visibility setting and save playlists made by others.
    \item No one other than the playlist owner can modify their original playlist.
    \item A user can make a copy of another user's playlist if they want to add their own modifications without manually rebuilding the said playlist.
    \item The playlist system should be smooth and support all presented use cases.
\end{itemize}

\subsection{1.5 Vision Statement} 
For music enthusiasts who seek hassle-free playlist creation, searching and sharing, with precise control over which parts of songs are heard in a playlist. Our playlist manager supports track start/end markers as metadata and allows building playlists that play only the track parts you want. Unlike other media player vendors, it lets you set start/end markers for each individual track and combine them into quick playlists, skipping the undesired parts of your music library altogether.

\subsection{2.2 Scope of the Initial Release}
\begin{itemize}
    \item Accounts: creation, deletion, profile customisation, the ability to sign in.
    \item Playlists: creation, deletion, naming, renaming, visibility setting (public/private).
    \item Default track catalogue containing full metadata (e.g., artist, title, album, tags).
    \item Start/end marker metadata, stored per track and applied within a playlist.
    \item Track search with result filtering and paginated results.
    \item Stable ordering rules within a playlist.
    \item Adding and removing playlists from favourites.
    \item Playlist copying for personal modifications.
    \item The ability to set or remove a playlist image.
\end{itemize}

\section{Use Case Document}

\subsection{Add Songs to a Playlist [UC1]}
\textbf{Primary Actor:} User

\textbf{Preconditions:} 
\begin{itemize}
  \item The user must be logged into the system.
  \item The user must have a playlist that has already been created.
\end{itemize}

\textbf{Success Guarantee:}  
The song selected by the user is added to the playlist.

\textbf{Main Success Scenario:}  
The user opens a playlist that already exists. They select the ``add song'' option. They search for a song (see UC2) and select it. The system adds the song to the playlist and gives a confirmation that the song has successfully been added to the playlist.

\textbf{Extensions / Alternate Scenarios:}  
If a song has already been added to the playlist, the system displays ``This song has already been added to the playlist.''

\textbf{Miscellaneous / Open Issues:}  
The ability to add multiple songs at once.

\subsection{Search Tracks [UC2]}
\textbf{Primary Actor:} User

\textbf{Preconditions:}
\begin{itemize}
  \item None.
\end{itemize}

\textbf{Trigger:} User selects the ``search'' option.

\textbf{Main Success Scenario:}
\begin{enumerate}
  \item The user selects the ``search'' option.
  \item The user enters a search query (song name, artist, album, etc.).
  \item The system searches the music database.
  \item The system displays a list of matching songs.
  \item The user can view details of a selected song.
\end{enumerate}

\textbf{Success Guarantee:}  
The system returns relevant songs that match the search criteria.

\textbf{Extensions / Alternate Scenarios:}
\begin{itemize}
  \item If no results are found, the system displays ``No songs found.''
  \item If the search query is empty, the system prompts the user to enter a valid search term.
\end{itemize}

\subsection{Set Start/End Times for Tracks in a Playlist [UC3]}
\textbf{Primary Actor:} User who is a playlist owner

\textbf{Preconditions:}
\begin{itemize}
  \item User is signed in.
  \item User owns the playlist.
  \item The playlist contains at least one track.
\end{itemize}

\textbf{Main Success Scenario:}
\enlargethispage{\baselineskip}
\begin{enumerate}
  \item The user selects a track inside their playlist.
  \item The user provides start and end times (e.g., 00{:}50--01{:}35).
  \item The system checks and confirms the times.
  \item The system saves the metadata and confirms the operation.
  \item The user sees the updated track in the playlist with correct time markers.
\end{enumerate}

\textbf{Success Guarantee:}  
The system stores the marker data for a given track in the context of the playlist.

\textbf{Extensions / Alternate Scenarios:}
\begin{itemize}
  \item The provided start/end times are invalid (e.g., exceed the song length or are negative).
  \item The user cancels the operation; the time markers are not updated.
  \item The track is removed during the operation; the system detects the change and aborts the operation.
\end{itemize}

\subsection{Brief Use Cases}

\begin{itemize}
    \item \textbf{Brief Use Case 1 [UC4] -- Delete Playlist:} The user is able to permanently delete a playlist.
    \item \textbf{Brief Use Case 2 [UC5] -- Share Playlists:} The user is able to generate a link to share their playlist with friends, family and the community.
    \item \textbf{Brief Use Case 3 [UC6] -- Save Playlists:} The user is able to save any public playlist to their favourites, or remove it from favourites.
    \item \textbf{Brief Use Case 4 [UC7] -- Remove Song from a Playlist:} The user can remove a song from their playlist.
\end{itemize}

\section{Project Estimation and Prioritization}

Below is a table with prioritized use cases. (Fill in time estimates and priorities as needed.)

\begin{table}[h]
  \centering
  \begin{tabular}{|c|c|c|}
    \hline
    Use Case & Time Estimation & Priority \\ \hline
    UC1 & 2 weeks & 1 \\ \hline
    UC2 & 2 weeks & 1 \\ \hline
    UC3 & 2 weeks & 2 \\ \hline
  \end{tabular}
\end{table}

\section{Project Plan and Schedule}

Inception: Use cases and vision scope

Elaboration: UC2

Construction: UC1, UC4, UC5

Transition:

\begin{longtable}{|c|c|c|c|c|c|}
    \hline
    \textbf{Week} & \textbf{Use Cases} & \textbf{Expected Hours} & \textbf{P.O. (Initials)} & \textbf{Sprint} & \textbf{Consultation} \\
    \hline
    1 & None &  &  & 1 & \textbf{A1 Presentation} \\
    \hline
    2 & UC1, Android skeleton &  &  & 1 & Model Drafts \\
    \hline
    3 & UC2, UC3 &  &  & 2 & \textbf{A2 Presentation} \\
    \hline
    4 & UC4, UC5, UC6 &  &  & 2 & Dev support \\
    \hline
    5 & UC7, UC8 &  &  & 2 & Dev support \\
    \hline
    6 & UC9, UC10 &  &  & 3 & \textbf{A3 Presentation} \\
    \hline
\end{longtable}


\section{Project skeleton}


\url{https://github.com/jok10/HBV501G_Hopur25}

\begin{itemize}
  \item jok90, jok10 -- Jóhann Kjartansson
  \item Loftur5 -- Loftur Páll Eiríksson
  \item masderni -- Andri Már Sigurðsson
  \item MountResonance -- Valentin Grechikhin
\end{itemize}



\end{document}
